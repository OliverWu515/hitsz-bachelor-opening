% !TEX root = ../main.tex

\chapter{引用参考文献}[Cite reference]

\section{引言}[Introduction]

\lipsum[1]

\section{参考文献引用方法}[How to cite the reference]

\sindex[china]{du!段誉}引文标注遵照GB/T7714-2005,采用顺序编码制。正文中引用文献的标示应置于所引内容最后一个字的右上角,所引文献编号用阿拉伯数字置于方括号“[ ]”中,用小4号字体的上角标。要求:

(1)引用单篇文献时,如“二次铣削\cite{ren2010}”。

(2)同一处引用多篇文献时,各篇文献的序号在方括号内全部列出,各序号间用“,”,如
遇连续序号,可标注讫序号。如,…形成了多种数学模型\cite{Gravagne2003,ren2010}…
注意此处添加\cs{inlinecite}中文空格\inlinecite{Gravagne2003,ren2010},可以在cfg文件中修改空格类型。

(3)多次引用同一文献时,在文献序号的“[ ]”后标注引文页码。如,…间质细胞CAMP含量
测定\cite[100-197]{Gravagne2003}…。…含量测定方法规定
\cite[92]{Gravagne2003}…。

(4)当提及的参考文献为文中直接说明时,则用小4号字与正文排齐,如“由文献\inlinecite{webster2010}可知”

(5)多\cite{liu2016}引\cite{fu2018}用\cite{zhai2015}一\cite{yao2015}些\cite{jones2006}参\cite{mcmahan2005}考\cite{jones2004}文献以生成附录参考文献。

\section{本章小结}[Brief summary]

\lipsum[1]
